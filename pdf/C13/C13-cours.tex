\PassOptionsToPackage{dvipsnames,table}{xcolor}
\documentclass[10pt]{beamer}
\usetheme[options]{Madrid} 
%\usetheme{Luebeck}
\usepackage{../../../latex/Cours}
\setbeamertemplate{items}[ball]
\begin{document}
	\input{\detokenize{../../../latex/MacrosCours.tex}}
	\setcounter{numchap}{13}


\newcommand{\Arch}{\cnum Les Booléens}

% Transistor et booléens
\begin{frame}
	\mframe{\Arch}
	\begin{block}{Remarques :}
		\begin{itemize}[label=\textbullet]
			\item<1-> Le composant de base des ordinateurs est le \textit{transistor}, un composant électronique ne pouvant être que dans deux états. Soit il laisse passer le courant (état \textcolor{red}{1}), soit il ne le laisse pas passer (état \textcolor{red}{0}).
			\item<2-> Toutes les données représentées dans un ordinateur le sont donc sous forme de 0 et de 1.
			\item<3-> Dès les années 1850, dans des travaux sur la logique, le mathématicien britannique Georges Boole avait travaillé sur des variables ne pouvant prendre que deux valeurs 0  ou 1.
			\item<4-> On appelle, ces variables des \textcolor{red}{booléens}. On définit trois opérations de base que nous allons détailler sur les booléens : le \textcolor{red}{non}, le \textcolor{red}{et} et le \textcolor{red}{ou}.
		\end{itemize}
	\end{block}
\end{frame}



% Opérateur non
\begin{frame}
	\mframe{\Arch}
	\begin{alertblock}{Opérateur \textbf{non}}
		\begin{itemize}[label=\textbullet]
			\item<1-> Inverse la valeur de l'entrée
			\item<2-> Symbole électronique
			      \begin{center}
				     \includegraphics[scale=0.25]{../../T01_TypesBase/T1.4_Booleens/data/porteNOT_USA_europe}
			      \end{center}
			\item<3-> Table de vérité
			      \begin{center}
				      \begin{tabular}{|>{\color{blue}}c|>{\color{red}}c|}
					      \hline
					      Entrée & Sortie \\
					      \hline
					      0      & 1      \\
					      \hline
					      1      & 0      \\
					      \hline
				      \end{tabular}
			      \end{center}
		\end{itemize}
	\end{alertblock}
\end{frame}


% Opérateur et
\begin{frame}
	\mframe{\Arch}
	\begin{alertblock}{Opérateur \textbf{et}}
		\begin{itemize}[label=\textbullet]
			\item<1-> Vaut 1 lorsque les \textit{deux} entrées valent un
			\item<2-> Symbole électronique
			      \begin{center}
				      \begin{tabularx}{0.8\textwidth}{Y|Y}
					      \begin{circuitikz} \draw
						      node[american and port](t1) {}
						      ;\end{circuitikz} &
					      \begin{circuitikz} \draw
						      node[european and port](t1) {}
						      ;\end{circuitikz}            \\
					      Américain                 & Européen \\
				      \end{tabularx}
			      \end{center}
			\item<3-> Table de vérité
			      \begin{center}
				      \begin{tabular}{|>{\color{blue}}c|>{\color{blue}}c|>{\color{red}}c|}
					      \hline
					      Entrée 1 & Entrée 2 & Sortie \\
					      \hline
					      0        & 0        & 0      \\
					      \hline
					      1        & 0        & 0      \\
					      \hline
					      0        & 1        & 0      \\
					      \hline
					      1        & 1        & 1      \\
					      \hline
				      \end{tabular}
			      \end{center}
		\end{itemize}
	\end{alertblock}
\end{frame}


% Opérateur NAD
\begin{frame}
	\mframe{\Arch}
	\begin{alertblock}{Opérateur \textbf{NAD}}
	Deux autres portes logiques sont fondamentales et bien que pouvant être 
	construire à partir de OR, AND et NOT ont leur propre symbole :
		\begin{itemize}[label=\textbullet]
			\item<1-> La porte NAND qui vaut 0 seulement lorsque les deux 
			entrées valent 1. C'est la porte "\textbf{NON ET}"
			\item<2-> Symbole électronique
			      \begin{center}
\includegraphics[scale=0.2]{../../T01_TypesBase/T1.4_Booleens/data/porteNAND_USA_Europe}
			      \end{center}
			\item<3-> Table de vérité
			      \begin{center}
				      \begin{tabular}{|>{\color{blue}}c|>{\color{blue}}c|>{\color{red}}c|}
					      \hline
					      Entrée 1 & Entrée 2 & Sortie \\
					      \hline
					      0        & 0        & 1      \\
					      \hline
					      0        & 1        & 1      \\
					      \hline
					      1        & 0        & 1      \\
					      \hline
					      1        & 1        & 0      \\
					      \hline
				      \end{tabular}
			      \end{center}
		\end{itemize}
	\end{alertblock}
\end{frame}

% Autres portes : NAND et XOR
\begin{frame}
	\mframe{\Arch}
	\begin{block}{Opérateur XOR}
		\begin{itemize}[label=\textbullet]
			\item<2-> La porte \textbf{XOR} qui vaut 1 lorsque l'une des 
			entrées vaut un 
			mais pas les deux à la fois. C'est \textbf{le ou exclusif}.
			\item<3-> Symbole électronique
			\begin{center}
				\includegraphics[scale=0.15]{../../T01_TypesBase/T1.4_Booleens/data/porteXOR_USA_Europe}
			\end{center}
			\item<4-> Table de vérité
			\begin{center}
				\begin{tabular}{|>{\color{blue}}c|>{\color{blue}}c|>{\color{red}}c|}
					\hline
					Entrée 1 & Entrée 2 & Sortie \\
					\hline
					0        & 0        & 0      \\
					\hline
					0        & 1        & 1      \\
					\hline
					1        & 0        & 1      \\
					\hline
					1        & 1        & 0      \\
					\hline
				\end{tabular}
			\end{center}
		\end{itemize}
	\end{block}
\end{frame}

% Autres portes : NAND et XOR
\begin{frame}
	\mframe{\Arch}
	\begin{block}{Autres portes logiques}
		Deux autres portes logiques sont fondamentales et bien que pouvant être construire à partir de OR, AND et NOT ont leur propre symbole :
		\begin{itemize}[label=\textbullet]
			\item<1-> La porte \textbf{XOR} qui vaut 1 lorsque l'une des 
			entrées vaut un 
			mais pas les deux à la fois. C'est \textbf{le ou exclusif}.
			\item<2-> Symbole électronique
			\begin{center}
				\includegraphics[scale=0.15]{../../T01_TypesBase/T1.4_Booleens/data/porteXOR_USA_Europe}
			\end{center}
			\item<3-> Table de vérité
			\begin{center}
				\begin{tabular}{|>{\color{blue}}c|>{\color{blue}}c|>{\color{red}}c|}
					\hline
					Entrée 1 & Entrée 2 & Sortie \\
					\hline
					0        & 0        & 0      \\
					\hline
					1        & 0        & 1      \\
					\hline
					0        & 1        & 1      \\
					\hline
					1        & 1        & 0      \\
					\hline
				\end{tabular}
			\end{center}
		\end{itemize}
	\end{block}
\end{frame}

% Autres portes : NOR
\begin{frame}
	\mframe{\Arch}
	\begin{block}{Opérateur NOR}
		\begin{itemize}[label=\textbullet]
			\item<1-> La porte NOR qui vaut 1 seulement lorsque les deux 
			entrées valent 0. C'est la porte "NON OU"
			\item<2-> Symbole électronique
			\begin{center}
				\includegraphics[scale=0.15]{../../T01_TypesBase/T1.4_Booleens/data/porteNOR_USA_Europe}
			\end{center}
			\item<3-> Table de vérité
			\begin{center}
				\begin{tabular}{|>{\color{blue}}c|>{\color{blue}}c|>{\color{red}}c|}
					\hline
					Entrée 1 & Entrée 2 & Sortie \\
					\hline
					0        & 0        & 1      \\
					\hline
					0        & 1        & 0      \\
					\hline
					1        & 0        & 0      \\
					\hline
					1        & 1        & 0      \\
					\hline
				\end{tabular}
			\end{center}
		\end{itemize}
	\end{block}
\end{frame}

%Python et les booléens
\begin{frame}
	\mframe{\Arch}
	\begin{block}{Python et les booléens}
		\begin{itemize}[label=\textbullet]
			\item<1-> Python possède le type de variable booléen, les deux valeurs possibles sont : \texttt{True} et \texttt{False}.
			\item<2-> L'opération \textbf{non} s'obtient à l'aide de \texttt{not}
			\item<3-> L'opération \textbf{et} s'obtient à l'aide de \texttt{and}
			\item<4-> L'opération \textbf{ou} s'obtient à l'aide de \texttt{or}
			\item<5-> Les booléens de python peuvent donc être notamment des résultats de test de condition.
		\end{itemize}
	\end{block}
	\onslide<6->{
		\begin{exampleblock}{Exemple}
			\texttt{\# Définit une variable booléen ok  qui vaut vrai} \\
			\texttt{\# lorsque au moins 2 des 3 variables a,b et c sont égales}\\
			\onslide<7->{\texttt{ok=(a==b) or (a==c) or (b==c)}}
		\end{exampleblock}}
\end{frame}

% Combinaison et réalisation d'opérations
\begin{frame}
	\mframe{\Arch}
	\begin{alertblock}{Circuit logique}
		\begin{itemize}[label=\textbullet]
			\item<1-> En combinant ces portes logiques, on réalise des circuits 
			logiques permettant d'effectuer des opérations (additions, 
			soustractions, comparaison, ...) sur les données stockées dans 
			l'ordinateur.
			\item<2-> Exemple : additionneur
			\includegraphics[scale=0.15]{../../T01_TypesBase/T1.4_Booleens/portes_logiques/additionneur3.png}
		\end{itemize}
	\end{alertblock}
\end{frame}


\end{document}